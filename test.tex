\documentclass[10pt, article, natbib]{IEEEtran}
\IEEEoverridecommandlockouts
\usepackage[utf8]{inputenc}
\usepackage[spanish]{babel}
\usepackage{fancyhdr}
\usepackage{mathtools}
\usepackage{tikz}

\pagestyle{fancy}
\fancyhf{Universidad Nacional de Costa Rica}
\rhead{\thepage}
\lhead{Proyecto \# 1}
\rfoot{EIF-212 Sistemas Operativos}
\lfoot{I-2023}

\def\changemargin#1#2{\list{}{\rightmargin#2\leftmargin#1}\item[]}
\let\endchangemargin=\endlist

\DeclarePairedDelimiter\ceil{\lceil}{\rceil}
\DeclarePairedDelimiter\floor{\lfloor}{\rfloor}
\makeatletter
\renewcommand*\l@section{\@dottedtocline{1}{1.5em}{2.3em}}
\makeatother
\begin{document}

\begin{titlepage}
	\includegraphics[width=0.2\textwidth]{./logo-UNA blanco.png}      
   	\begin{changemargin}{4.5cm}{0cm}
       	\textbf{\Huge Contando Bytes}

       	\vspace{0.2cm}
       	\LARGE Proyecto \# 1
            
       	\vspace{3cm}
		\Large
       	Jorge Durán Campos \\ 
       	Luis Antonio Montes de Oca Ruiz \\ 
       	Diego Quirós Artiñano \\ 

       	\vspace{3cm}
       
		EIF-212 Sistemas Operativos \\
       	Profesor Eddy Ramirez Jiménez \\
		       	
       	\vspace{3cm}
       	\today
	\end{changemargin}
\end{titlepage}

\onecolumn
% Índices
\pagenumbering{roman}
    % Contenido
    \renewcommand{\contentsname}{\large Índice \\ \hrulefill}
\tableofcontents
\setcounter{tocdepth}{2}
\newpage
%     Figuras
 \renewcommand{\listfigurename}{\large Índice de fíguras \\ \hrulefill}
 \listoffigures
 \newpage
     % Tablas
 \renewcommand{\listtablename}{\large Índice de tablas \\ \hrulefill}
 \listoftables
 \newpage

% Cuerpo
\pagenumbering{arabic}

\section{Resumen Ejecutivo}

\newpage
\twocolumn
\section{Introducción}
Este documento tiene como objetivo servir como documentación del primer proyecto del curso de sistemas operativos. Este consta de un resumen ejecutivo en el cual se resume el proyecto junto a su solución y resultados, un marco teórico donde se describen aspectos relacionados al proyecto, como el leguaje y las bibliotecas, una descripción de la solución implementada al problema planteado en el enunciado del proyecto, los resultados de las pruebas de la solución, las conclusiones donde se analizan los resultados y el proyecto en general, y por ultimo se presentan los aprendizajes obtenidos al realizar el proyecto. Así mismo, el proyecto se basa en realizar un programa lector de cualquier tipo de archivo, sea un video, un texto o demás, si este se sobrepasa de la cantidad de memoria disponible se realiza la lectura poco a poco, este archivo es leído de manera binaria usando hilos “productores” que van byte por byte, estos bytes se entregan a los hilos “consumidores”, y van a colocar cada uno de los 256 posibles valores de bytes en un arreglo de 256 espacios, en el que cada índice representa la cantidad de repeticiones de un byte en particular.

\section{Marco teórico}
El lenguaje utilizado en este proyecto es C, el cual se creó a principios de los años 1970 por Dennis M. Ritchie en Bell Laboratories. Este fue diseñado como un lenguaje minimalista para la creación de sistemas operativos para minicomputadoras, las cuales eran computadoras más baratas y menos potentes que una supercomputadora, pero más caras y potentes que una computadora personal. El principal motivo fue el deseo de migrar el kernel del sistema pronto a ser terminado, UNIX, a un lenguaje de alto nivel, teniendo las mismas funciones, pero con menos líneas de código. C se basó en CPL, o Combined Programming Language, por sus siglas en inglés, el cual a su vez sirvió de base para el lenguaje B. De este, Ritchie reescribió varias funciones de CPL para crear C y después reescribió UNIX en este nuevo lenguaje.\\

Desde 1977 hasta 1979 ocurrieron distintos cambios en el lenguaje, y durante este tiempo se publicó un libro que sirve como manual para el lenguaje, titulado "The C Programming Language", publicado en 1978 por Ritchie y Brian W. Kernighan. Cinco años después, se estandarizó C en el American National Standards Institute, y desde ese momento, al lenguaje se le refiere como ANSI Standard C. De C salieron varios lenguajes derivados, tales como Objective C y C++. Además, también surgió Java, el cual se creó como un lenguaje que simplifica C.\\

Continuando con los recursos utilizados en este proyecto, se utilizó la librería pthread. Esta es una librería de POSIX la cual es un estándar para el uso de hilos (threads), en C y C++, los cuales permiten un flujo de procesos de manera concurrente. El uso de estos hilos es mas efectivo en procesadores con múltiples núcleos, ya que se pueden asignar los procesos a distintos núcleos, haciendo la ejecución más rápido.\\

Además, se escogió CLion como el IDE preferido para este proyecto. Este IDE es de la empresa JetBrains y fue lanzado al mercado en 2014 con herramientas que ayudan a la creación de código, tales como refactoring, generación de código para sets/gets, terminación de código y arreglos rápidos del código escrito.\\

Finalmente, ya que se mencionó tanto POSIX como UNIX se va a brindar información sobre estos. Portable Operating System Interface for UNIX, o POSIX, son estándares establecidos por la IEEE y publicados por la ANSI e ISO (International Organization for Standardization), estos estándares permiten el desarrollo de código universal, para que pueda correr en todos los sistemas operativos que implementen POSIX, tales como macOS o Ubuntu, la mayoría de los sistemas basados en UNIX cumplen con POSIX. UNIX es un sistema operativo que fue creado para brindar a programadores funciones simples pero potentes, y que permitiera el uso de múltiples usuarios y de multi tarea, este se compone de tres partes, el kernel, los archivos de configuración de sistema y los programas.\\

\section{Descripción de la solución}

\section{Resultados de pruebas}

\section{Conclusiones}

\section{Aprendizajes}
\cite{threadsInCEducative}

\newpage
\onecolumn
\bibliographystyle{ieeetr} 
\bibliography{ref.bib}

\end{document}
